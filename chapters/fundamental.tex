\section{The Fundamental Theory of Studying}

\subsection{Origin Story}
The story is painful, but true. During my physics bachelor I had the lecture `Numerical Methods for Physicists'. I liked it, got a 5.25 and was happy. Of course, a better grade would have been preferable, but I probably had not done enough practising there either... Then I applied for the teaching assistant jobs and got one for NumPhys. It was a blast and I had fun teaching. Even if it was online via Zoom and Goodnotes and not in the huuuge lecture room NO C60 in ETH Zentrum. Two years later, I am sitting in the `Numerical Methods for Computer Scientist' lecture. Or rather watch the designated videos because the lecturer can't be bothered to do it in person. Anyways, I get the bonus of 0.25 by solving some tasks, do — again — too few during the semester — and go to the exam. The big problem now was that I did not practise, study, learn (doesn't matter how you call it) the basics properly. SVD, Newtons method, Lagrange polynomials, ... all that stuff was familiar and I understood the concepts and knew when to use them. But how exactly, I didn't bother to really solidly know that. The fundaments of the subject were not built. I had enough time in the exam. It was very fair and easy actually. 13/15 multiple choice questions I could answer straightforward. But then in the text questions I hesitated at the definition of Newton Basis and got it wrong. Then we were asked about QR in Least Squares problems and also SVD. The questions were e a s y. But I wasn't sure. My code answers felt too long. Probably missed a simplification while deriving it. And in general, I just was not sure. Do they go into the normal equations, or straight into the squared expression? Don't we need to consider them squared such that a normal norm would not suffice? 
So many different questions because my understanding was shaky. Unstructured and brittle. The smallest disturbing factor and it all collapsed. This is how I gave away soo soo many free points. Simple definitions and principles were not known. In closed book exams especially, it is all about the fundamentals. If they are not solid, nothing of essence can be achieved. No slightly deviated tasks can be solved with assurance, no reasoning is correct to the roots. 

Thus, we need to organise subjects according to The Fundamental Theory of Studying. Set priorities during the semester (!)\ and the study phase accordingly. If your brain got all the necessary resources to be at ETH, with the right principles, you can achieve extraordinary grades and loads of subject success stories. So let's dive into it!

\newpage

\subsection{Theory: Basics}
Most important is the division of content. There are three main types of content: fundament, building blocks and decorations.

\subsubsection{Fundament}
In contrast to the two other types, this is just one thing. It is THE fundament. It is the low structural elements, upon which all the rest is built up. It consists of base ingredients, which, if combined and interleaved thoroughly, can build up an incredibly strong and robust structure. All the rest relies on this one single fundament. 

\textit{Example: When applying this theory to Analysis 2, we quickly see that the fundament consists of things like the Jacobian, the Hessian, Total Differential, multidimensional limits, differentiation in $\mathbb{R}^n$. These are no fancy sentences, no complicated procedures or examples. They are, what all those things need.}

\subsubsection{Building Blocks}
Any building would not look like much if there was just a fundament. So there really should be something on top of it. 
This is where the building blocks come in. They use ingredients from the fundament and create actually useful stuff with it. They generally are the main content of the subject we want to study. All the blocks have a rather high sense of relevance. It is not just one extremely difficult and edge case example covered in several minutes, but recurring and useful stuff. 
Some depend on others, some are standalone, but together they build the house.

\textit{Examples: There are theorems that use the definitions and conventions. Based on theorems we can build lemmas on top. Furthermore, there are examples based on all of that, and we have general procedures to solve certain tasks.}

\subsubsection{Decorations}
Generic houses are good, but not excellent. Once we have a SOLID fundament, and a well-structured core out of the building blocks, we can think about decorating it. Not earlier, just now. Because what would the longevity of a gold-plated ornament be, once a little earthquake happens and the building collapses on top? 
The robustness of such a house is right about 0. 
We need the decorations to really stand out. They are advanced stuff that is not likely to be in an exercise of an exam, but rather exquisite and interesting stuff nonetheless. Gems of the subject. It shall be noted, that insights can be gained by making decorations, that can be used in other parts. 
So don't learn them too early, but do if you got time in the end.

\textit{Examples: Proofs of theorems, four page examples, sidenotes.}


\subsection{How to Study - during the Semester}
\subsubsection{General Strategy}
What we want to do, is builduing up a strong and durable fundament with longevity. That is the main goal. Always. This is done by digging into the core ideas, pondering about them, reasearching about them, reading stuff, discussing with friends, writing clarification emails to the teaching assitents and going up to the lecturers in the breaks.

\subsubsection{Before the Lecture}
Can do: We might read the script and see what the lecture will be about. 

Must do: Read the quick-summary written about the last lecture. 

\subsubsection{During the Lecture}

\subsubsection{After the Lecture}

\subsection{How to Study - for an Exam}
First, figure out the fundament. What are its ingredients? Go through the lecture documents and find the recurring objects, important definitions, and general concepts. Directly write a card deck with them, and regularly study them every day; often.
And ask yourself questions about them. Where do those come from? Why like this and not in another way? Can I explain the purpose and meaning in three sentences? How is it used?

Then, go to the building blocks. What are the different topics covered. In every one of them, there are certain types of exercises that can be done. What are they? and how can one approach them? Figure this out and condense the principles. What are the main theorems and which corollaries are truly important? Of course, there is never actual certainty in answering these questions. There doesn't need to be. Just go with the gut feeling for now. Do \textit{something}. And then review them once there are problems coming up which can't be solved with the stuff that was already classified as building blocks.

\st{Then we go to the decorations.} No. Simply No. Doing too much reading work is not beneficial and not well-used study time according to The Fundamental Theory of Studying. This can be done during the semester or if certain concrete questions pop up. But only then.

Our actual next step is solving maaaany mock exams and  rouletting through the series exercises. Do this for a couple of days. Go back to the fundaments and the building blocks sporadically and check if they need adjustments. 

If you need breaks or alternation, okay, then you can do some decoration work. But be considerate of your time and use it wisely. \\


Structure the studying days, such that the \textbf{first and last} real session is \textbf{ALWAYS} without doubt \textbf{fundamental work}. It doesn't need to be for ages, but make a habit out of it. If you stick to this, the probability of facing a task and then loosing your footing, swimming, struggling or however you want to call it diminishes.\\

You will succeed. \\

You got this! \\


Now, smile and go studying. \smiley{}
